\documentclass[nobib]{tufte-handout}
%\usepackage{etex} \reserveinserts{36}
\usepackage{graphicx}
\usepackage{morefloats}
\usepackage{cite}
\setkeys{Gin}{width=\linewidth,totalheight=\textheight,keepaspectratio}
\graphicspath{{figures/}}

% The units package provides nice, non-stacked fractions and better spacing
% for units.
% \usepackage{units}

% The fancyvrb package lets us customize the formatting of verbatim
% environments.  We use a slightly smaller font.
\usepackage{fancyvrb}
\fvset{fontsize=\normalsize}

% Small sections of multiple columns
% \usepackage{multicol}

\hypersetup{colorlinks}

\newcommand{\minisec}[1]{\vspace{3pt} \noindent \emph{#1}\ }

\title{Inviting Darwin into protein foundation models}
\author{\href{http://matsen.fredhutch.org/}{Frederick A. Matsen}, Fred Hutchinson Cancer Center}


\begin{document}

\maketitle

\begin{abstract}
\noindent
Language models of protein sequences have tremendous promise for understanding and predicting the function of protein variants.
However, the current generation of models ignore evolutionary biology.
We have shown that by placing a deep neural network inside a classical mutation-selection model for antibodies, we obtain a model that does much better at functional prediction, as well as being more interpretable, 10x smaller, and 500x faster.
\end{abstract}

\newthought{Evolution is a process of mutation and selection.}
This simple statement is the foundation of modern biology.
High-throughput sequencing now delivers detailed information about the participants in this evolutionary competition, which pits the adaptive and innate immune systems versus viruses and bacteria.
I count myself lucky to be developing and applying algorithms to analyze sequence data from each of these four major players.
Here I will mostly focus on our most recent work on adaptive immunity, but will mention the other components after.

% %
% \begin{marginfigure}[-5.3cm]%
%   \hspace{-17pt}
%   \includegraphics[width=2.2in]{overview.png}%
%   \caption{\
%   The VDJ rearrangement process consists of random selection of V, D, and J genes, random trimming of their ends, then joining them with random nucleotides.
%   }%
%   \label{FIGvdj}%
% \end{marginfigure}

\newpage
\bibliography{research}
\bibliographystyle{tufte}

\end{document}


% ,matsen2012ubiquity
